\documentclass[
%     paper,
    manuscript,
%     twocolumn,
%     twoside,
%     revised,
  ]{geophysics}

% Remove once done
\usepackage[textsize=footnotesize, textwidth=2.8cm]{todonotes}

% Additional packages to geophysics.cls
\usepackage[UKenglish]{babel}
\usepackage[utf8]{inputenc}
\usepackage{lmodern}
\usepackage[T1]{fontenc}

\usepackage{amssymb, amsmath, amsfonts}
\usepackage{upquote}
\usepackage[strings]{underscore}
\usepackage{siunitx}                % SI conform commands (eg \num)
\usepackage{tabularx}
\usepackage{colortbl}
\usepackage{booktabs}
\usepackage{xspace}

\usepackage[pdftex, final]{hyperref}
% \usepackage[pdftex, hidelinks]{hyperref}
\hypersetup{allcolors=blue, allbordercolors={0 0 .5}, colorlinks=true}

% Figures
\DeclareGraphicsExtensions{.pdf,.png,.jpg}
\renewcommand{\figdir}{./figures}
\ifthenelse{\boolean{@twoc}}{%
  \newcommand{\cwidth}{250pt}}{%
  \newcommand{\cwidth}{240pt}}

\ifthenelse{\boolean{@twoc}}{%
  \newcommand{\acwidth}{250pt}%
  \newcommand{\tcwidth}{250pt}}{%
  \newcommand{\acwidth}{240pt}%
  \newcommand{\tcwidth}{312pt}}

% Own commands
\newcommand{\mr}[1]{\mathrm{#1}}
\newcommand{\emg}[2]{\texttt{emg#1#2}\xspace}
\newcommand{\empymod}{\texttt{empymod}\xspace}
\newcommand{\simpeg}{\texttt{SimPEG}\xspace}
\newcommand{\custem}{\texttt{custEM}\xspace}

% Remove once done
\newcommand{\itodo}[1]{\todo[color=blue!40!white, inline]{\sffamily #1}}
\newcommand{\mtodo}[1]{\todo[inline]{\sffamily #1}}
\newcommand{\etodo}[1]{\todo[color=blue!40!white]{#1}}
\newcommand{\dul}[1]{\underline{\underline{#1}}}
\newcommand{\ul}[1]{\underline{#1}}
\newcommand{\rmk}[1]{{\color{red}{#1}}}

\begin{document}

\title{Fast time-domain electromagnetic modelling in the frequency domain}

\renewcommand{\thefootnote}{\fnsymbol{footnote}}

\ms{}  % ARTICLE-ID

\address{
\footnotemark[1]TU Delft,
Building 23,
Stevinweg 1 / PO-box 5048,
2628 CN Delft;\\
\footnotemark[2]Shell Global Solutions International BV,
Grasweg 31, 1031 HW Amsterdam;\\
E-mail: \href{mailto:Dieter@Werthmuller.org}{Dieter@Werthmuller.org};\\
\textbf{Keywords}: Fourier transforms, CSEM, 3D modelling.
}


\author{%
Dieter Werthmüller\footnotemark[1], %            orcid: 0000-0002-8575-2484
Wim A.\ Mulder\footnotemark[1]\footnotemark[2], % orcid: 0000-0001-7020-9297
and Evert C.\ Slob\footnotemark[1] %             orcid: 0000-0002-4529-1134
}

\footer{}
\lefthead{Werthmüller et al.}
\righthead{Time-domain CSEM}

\maketitle

%%fakesection ===    ABSTRACT    ===
\begin{abstract}
%
Modelling three-dimensional time-domain controlled-source electromagnetic data
with frequency-domain solvers is an alternative to time-domain solvers. It
requires the computation of many frequencies if standard Fourier transforms are
used, which can make it prohibitively expensive in comparison with a
time-domain computation. The speed of time-domain modelling with
frequency-domain computation is defined by three key points: solver, method and
implementation of the Fourier transformation, and gridding.
%
The faster the frequency-domain solver, the faster time-domain modelling will
be. It is important that the solver is robust over a wide range of frequencies.
A three-dimensional, iterative, frequency-domain multigrid solver satisfies
this requirement, but any other robust solver will do.
%
The method should require as few frequencies as possible while remaining
robust. As the frequency range spans many orders of magnitude, the required
frequencies are ideally equally spaced on a logarithmic scale. The fast
frequency-to-time domain method uses either the digital linear filter method
or the logarithmic fast Fourier transform together with a careful selection of
frequencies and interpolation. This methodology requires typically 15 to 20
frequencies to cover a wide range of offsets in one-dimensional test models.
%
The gridding should be frequency-dependent, which is accomplished by making it
a function of skin depth. Optimising for the least number of required cells for
a given frequency should be combined with optimising for computational speed.
%
Looking carefully at these three points results in much smaller computation
times with speedup factors of ten or more over previous results. A
frequency-domain computation of time-domain data can therefore be competitive
to time-domain solvers if the required frequencies and the corresponding grids
are carefully chosen.
%
\end{abstract}

\section{Introduction}

The controlled-source electromagnetic (CSEM) method is one of the common
non-seismic tools in exploration geophysics, not only in hydrocarbon
exploration \citep{GEO.10.Constable}, but also in the search for sulfides
\citep{GRL.19.Gehrmann}, water \citep{GEO.05.Pedersen}, geothermal sources
\citep{WGC.15.Girard}, or geological purposes \citep{NAT.19.Johanson}. While
current sources with a few frequencies are used in the deep marine environment,
transient measurements are more common in the shallow marine environment and on
land \citep[e.g., ][]{GEO.07.Ziolkowski, SEG.07.Andreis, SEG.07.Avdeeva}. One
of the main reasons is the dominance of the airwave in shallow marine and
terrestrial measurements, which can be better separated in the time domain.
CSEM is usually divided into frequency- and time-domain methods, depending
whether the source signal is a continuous waveform, such as a sine, or a finite
waveform, such as a pseudo-random binary sequence (PRBS). A numerical
comparison of the two methods is given by \cite{GP.13.Conell}. Acquired CSEM
data are subsequently often analysed (modelled and inverted for) in their
respective domain, either the frequency or time domain. Modellers of layered
media usually exploit the horizontal shift-invariance by computing the
responses in the wavenumber-frequency domain followed by a 2D inverse spatial
Fourier transform, also called Hankel transform, to the space-frequency domain,
and a regular inverse Fourier transform if time-domain data are required
\citep[e.g., ][]{GEO.15.Hunziker}.

CSEM codes for 2D and 3D computations, on the other hand, mostly compute their
responses directly in the required domain, either frequency or time. There is a
wealth of 3D electromagnetic codes, with a first boom in the 1990's, with a
good state of the art provided by \cite{B.SEG.99.Oristaglio}. A second wave
occurred during the CSEM boom in the hydrocarbon industry in the early 21th
century, see, e.g., \cite{SG.05.Avdeev} and \cite{SG.10.Borner} for
comprehensive overviews. Recent publications span the width of numerical
methods from integral equation codes \citep{MGS.17.Kruglyakov} and differential
equation methods such as finite differences \citep{CAG.13.Sommer}, finite
elements \citep{GJI.13.Grayver}, and finite volumes \citep{GEO.14.Jahandari}.
Besides the different methods there are also many different types of
discretisation. Commonly used, besides rectilinear Yee grids
\citep{IEEE.66.Yee} as used by our code, are Lebedev grids
\citep{CMMP.64.Lebedev} but also unstructured grids using, for instance,
tetrahedral \citep{CAG.17.Cai}. The computational requirement to model CSEM
data also depends heavily on the technique which is used to solve the system,
where there are direct solvers \citep{GEO.15.Grayver}, indirect solvers
\citep{GJI.15.Jaysaval}, and more recently also hybrid solvers
\citep{GEO.18.Liu}.

Under certain conditions it can be competitive to model time-domain data with a
frequency-domain code, as shown by \cite{GEO.08.Mulder}. The conditions are: a
sufficiently powerful solver, appropriate frequency selection and
interpolation, and an automated gridding, for which they used the
multi-frequency CSEM approach presented by \cite{GEO.07.Plessix}. We build upon
these results but improve the run time from hours to minutes. The main reasons
for this significant speed-up are an adaptive, frequency-dependent gridding
scheme that minimizes the required cells in each dimension, and a logarithmic
Fourier transform such as digital linear filters \citep[DLF, ][]{GP.71.Ghosh}
or the logarithmic fast Fourier transform \citep[FFTLog, ][]{RAS.00.Hamilton}
to go from the frequency to the time domain. The latter makes it also possible
to only use the imaginary part of the frequency-domain response, which has
advantages when it comes to interpolation.

In the next section, we briefly review the methodology as introduced by
\cite{GEO.07.Plessix} and \cite{GEO.08.Mulder} and highlight their advantages
and shortcomings. This is followed by an outline of our changes to the Fourier
transform and the adaptive gridding. Finally, we demonstrate the efficiency of
the approach with some numerical results.

\section{Motivation}

Being able to model CSEM data both in the frequency-domain and in the
time-domain can be desirable, as both domains have advantages and
disadvantages. One way to achieve this is to implement Maxwell's equation in
both domains, as it is done, for instance, in \simpeg \citep{CAG.15.Cockett}.
Another approach is to have Maxwell's equation only implemented in one domain,
and use Fourier transforms to go to the other. However, this approach can be
costly, as many frequencies over a wide range are required to go from the time
domain to the frequency domain, or many times over a wide range for the
opposite direction. We present a methodology which significantly reduces the
amount and range of the required frequencies, translating into a significant
reduction in computation time. The required computation grids for low
frequencies (in our case in the order of 0.001\,Hz) and high frequencies
(around 100\,Hz) are hugely different. Low frequencies can be computed on a
coarser grid, but they require a much larger domain in order to avoid boundary
effects. High frequencies, on the other hand, require denser gridding, but they
are much more limited in reach. An adaptive gridding scheme is therefore
required. We build our approach upon \cite{GEO.07.Plessix}, who presented such
an adaptive gridding for multi-frequency (and multi-source) CSEM modelling.

An adaptive grid in the frequency domain is naturally based on the skin depth,
which is the distance after which the amplitude of the electromagnetic field is
decayed by $1/e\approx 37\,\%$. The skin depth $\delta$ is a function of
conductivity and frequency, and for the diffusive approximation of Maxwell's
equation in an isotropic, homogeneous medium is given by \citep[e.g.,][
equation~1.53]{B.SEG.88.Ward}
%
\begin{equation}
  \delta = \sqrt{\frac{2}{\omega\mu\sigma}}
  \ \approx \
  % \frac{503.3}{\sqrt{f\sigma}} \, ,
  % 503.3(f\sigma)^{-1/2} \, ,
  503.3/\sqrt{f\sigma} \, ,
  \label{eq:skindepth}
\end{equation}
%
where $\sigma$ is conductivity (S/m), $\omega=2\pi f$ is angular frequency of
frequency $f$ (Hz), and $\mu$ is magnetic permeability (H/m). The approximation
is obtained by using the free-space value of magnetic permeability,
$\mu_0=4\pi\times10^{-7}\,$H/m.

\cite{GEO.07.Plessix} define the minimum cell width $\Delta_\mathrm{min}$ as a
fraction of the minimum skin depth $\delta_\mathrm{min}$, where
$\delta_\mathrm{min}$ should be 2--3 times bigger than $\Delta_\mathrm{min}$.
The cells have to be smallest around the source; in the marine case, the
minimum skin depth is therefore defined by the conductivity of seawater.
However, this yields quite large cells for low frequencies, so special care has
to be taken around the source by defining a maximum allowed
$\Delta_\mathrm{min}$. The grid dimension, on the other hand, is defined as a
function of the skin depth for the average conductivity of the background,
$\delta_\mathrm{ave}$. They use four times $\delta_\mathrm{ave}$ for the $x$-,
$y$-, and downward $z$-directions, and a fixed 50\,km for the upward
$z$-direction to account for the airwave. To reduce the number of cells, it is
desirable to introduce stretching, at least in the boundary zone outside of the
area where source and receivers are located.

\cite{GEO.08.Mulder} provide a computational complexity analysis of various
methods to model transient electromagnetic responses directly in the
time-domain, and compare it to the computation of transient EM responses in the
frequency-domain with a subsequent Fourier transform. They conclude the review
by stating \emph{\guillemotleft Although it remains to be seen which of the
four methods requires the least computer time for a given accuracy, the
frequency-domain approach appears to be attractive.\guillemotright} Their
approach is to minimize the computation time by having, in addition to the just
introduced adaptive gridding, an adaptive frequency selection scheme. This
scheme starts with computing the responses for a set of just five frequencies,
regularly sampled on a log-scale, from minimum to maximum required frequency.
All the other frequencies are interpolated with a shape-preserving
piecewise-cubic Hermite interpolation \citep[PCHIP, ][]{SIAM.80.Fritsch}.
Testing the stability of the obtained response by removing a single
frequency-value at a time their scheme decides if more frequencies in-between
the already computed ones are required. In this way frequencies are only added
if required, hence if certain criteria of response stability are not met. While
this method is good and effective for a single offset, it loses all its
advantages if one tries to compute different offsets within one computation, as
each offset requires a different set of adaptive frequencies. Additionally, it
hampers the parallelization over frequencies.

We present improvements to both the adaptive gridding and the transform from
frequency domain to time domain, which makes time-domain modelling with a
frequency-domain code even more competitive in comparison with time-domain
codes. It is important to note that while we show the transform from frequency
to time domain, the conclusions regarding the Fourier transform can equally
well be applied to the transform from time to frequency domain data for a code
that solves Maxwell's equations in the time domain. A variation of our method
was implemented, for instance, by \custem \citep{GEO.19.Rochlitz} to obtain
frequency-domain data from time-domain computations (Rochlitz, private
communication, 2020).

\section{Methodology}

For the numerical computations we use the open-source (Apache License 2.0) code
\emg3d \citep{JOSS.19.Werthmuller}, a multigrid solver based on
\cite{GP.06.Mulder}, which can be used as a preconditioner for Krylov subspace
solvers or as a solver on its own. The multigrid approach works fine for the
diffusive approximation of Maxwell's equation, which assumes that
$\omega\varepsilon \ll \sigma$, where $\varepsilon$ is electric permittivity
(F/m). The remaining system to solve in the frequency domain is then given by
the second-order differential equation of the electric field,
%
\begin{equation}
    \mathrm{i}\omega\sigma \mathbf{E} +
    \nabla \times \mu^{-1} \nabla \times \mathbf{E}
    = -\mathrm{i}\omega\mathbf{J}_\mathrm{s} \, ,
  \label{eq:maxwell}
\end{equation}
%
where $\mathbf{E}$ is the electric field (V/m) and $\mathbf{J}_\mathrm{s}$ the
current source (A/m$^2$); time dependence is $\exp(\mathrm{i}\omega t)$. The
standard multigrid approach fails for severe stretching or strong anisotropy,
for which \emg3d has implemented known improvements such as line-relaxation and
semicoarsening \citep{ECCFD.06.Jonsthovel}, with a non-standard Cholesky
decomposition to speed up the computation \citep{GEO.08.Mulder}. One of the big
advantages of the multigrid method is that it scales linearly (optimal) with
the grid size in both CPU and RAM usage \citep{B.Springer.20.Mulder}. This
makes it feasible to run even big models on standard computers, without the
need for big clusters. All the examples in this article are run on a laptop
with an i7-6600U\,CPU\,@\,2.6\,GHz (x4) and 16\,GB of memory, using Ubuntu
18.04 and Python 3.7. For the semi-analytical computations of layered models we
use the open-source (Apache License 2.0) code \empymod
\citep{GEO.17.Werthmuller}.

\section{Frequency selection}

An important factor in terms of speed and accuracy for time-domain responses
obtained from frequency-domain computations is the selection of the required
frequencies. The fewer frequencies required, the quicker we obtain the
time-domain result. We decided to use a regular spacing of frequencies on a
log-scale, rather than an adaptive scheme. This approach is favourable if a
wide range of offsets is needed, as the required frequencies change with offset
and an adaptive frequency selection is therefore often tailored to a single
offset. Also, it allows for straightforward parallelization over frequencies,
which is not completely possible with an adaptive scheme.

For the actual transform we use either the digital linear filter (DLF) method
or the logarithmic fast Fourier transform (FFTLog). The DLF method was
introduced to geophysics by \cite{GP.71.Ghosh}, and is arguably the most common
method in EM geophysics for both its simplicity and its speed. It is
implemented for the Hankel and Fourier transforms in most EM modelling codes,
e.g., \cite{GEO.09.Key}. A simple tool to design digital linear filters was
recently presented by \cite{GEO.19.Werthmuller}, together with a comprehensive
overview of the history and development of DLF in geophysics. FFTLog,
introduced by \cite{RAS.00.Hamilton}, is another transform algorithm which
proved to be powerful for the frequency-to-time domain transformation of EM
responses, e.g., \cite{INT.14.Werthmuller}. In our tests they are both about
equal in speed and accuracy. DLF requires a wider range and many more
frequencies than the FFTLog. Both methods share some important characteristics
in comparison with the standard FFT: the required input frequencies are equally
spaced on a logarithmic scale, and they only require either the real or the
imaginary part of the frequency-domain response. We can take advantage of that
by using only the imaginary part of the frequency-domain response. The
imaginary part goes to zero when the frequency goes to zero or to infinity,
with the advantage that knowing the endpoints makes it possible to convert the
extrapolation of missing frequencies into interpolation.

There are three key-parameters that have to be defined: Minimum and maximum
frequency ($f_\mr{min}, f_\mr{max}$) and the number of frequencies per decade
($n_\mr{dec}^f$). Trial-and-error with a 3D code is very time-intensive.
However, a simplified, layered model for the required survey setup and a fast
1D modeller makes it possible to estimate these parameters easily. To this end,
we created a graphical, interactive tool using \empymod, as shown in
Figure~\ref{fig:GUI-FFTLog-lin}.
%
\plot{GUI-FFTLog-lin}{width=\tcwidth}{
  Interactive frequency selection for a user-provided layered model (the shown
  model parameters are described in the text). This example shows the impulse
  response at an offset of 5\,km, for which it uses FFTLog from 0.001\,Hz to
  10\,Hz with five frequencies per decade. Note that the error is clipped for
  values smaller than 0.01\,\% and bigger than 100\,\%.
}
%
The example model is a marine scenario with 1\,km water depth of resistivity
$\rho=0.3\,\Omega$\,m ($\rho = \sigma^{-1}$) and a 100\,m thick target of
100\,$\Omega$\,m at 1\,km below the seafloor in a background of 1\,$\Omega$\,m.
The source is 50\,m above the seafloor, the receiver is on the seafloor, and
the response is the inline $x$-directed $E$-field. The left subplot shows the
imaginary part in the frequency domain and the right plot the corresponding
impulse response in the time domain. The red lines are the precise results
obtained from \empymod. The blue circles indicate the actually computed
responses and the black dots the frequencies which are interpolated or set to
zero. The resulting time-domain response has a relative error of less than
1\,\% everywhere except for very early times.

Figure \ref{fig:GUI-FFTLog-log} shows exactly the same on a logarithmic scale,
without the interactive GUI-related elements. It can be seen that with the
chosen frequencies the time-domain response starts to divert above about 100\,s
and below 0.3\,s. It also shows the oscillating high-frequency part, which is
hard to interpolate. Figure \ref{fig:GUI-DLF-lin} is the same as
\ref{fig:GUI-FFTLog-lin}, but transformed with the DLF method applying the
81-point sine-cosine filter from \cite{GEO.09.Key}. The same frequencies were
computed as in the FFTLog case and the missing ones interpolated. The error of
the corresponding time-domain response is comparable, so either FFTLog or DLF
can be used.
%
\multiplot{2}{GUI-FFTLog-log,GUI-DLF-lin}{width=\tcwidth}
{(a) Same as Figure \ref{fig:GUI-FFTLog-lin}, but on a logarithmic scale.
(b) Using DLF instead of FFTLog, but computing the same frequencies as for the
FFTLog. The resulting time domain response has comparable accuracy (compare the
error to the error in Figure \ref{fig:GUI-FFTLog-lin}).
}
%
In the above example we used 20 frequencies, but that many would not be
required for the shown offset of 5\,km, a few of the lower and higher
frequencies could be left out. However, with this frequency-selection we can
model a wide range of offsets. This is shown in Figure \ref{fig:multi-offset},
where the above parameters were used to model the responses at offsets
$r=1.5,3,6$, and 12\,km.
%
\plot{multi-offset}{width=\tcwidth}{
  Normalized (a) frequency- and (b) time-domain responses for offsets
  $r=1.5,3,6$, and 12\,km using the parameters defined in Figure
  \ref{fig:GUI-FFTLog-lin}. The coloured circles are the actually computed
  responses, the black dots are the responses which are set to zero or
  interpolated. The black curves are the analytical responses.
}
%

The shortest offset defines the highest required frequency, and the largest
offset the lowest required frequency. So the more one can restrict the
necessary offset range, the less frequencies are needed. Another important
factor is how to interpolate and extrapolate from the computed frequencies to
the frequencies required for the Fourier transform. For the FFTLog only
extrapolation for higher and lower frequencies is required. The EM response
becomes highly oscillatory for high frequencies, which makes it very hard to
extrapolate the response to frequencies $f>f_\mathrm{max}$. However, if
$f_\mathrm{max}$ is chosen judiciously, the importance of higher frequencies
for the Fourier transform can be neglected and we can set those responses to
zero. The extrapolation of frequencies $f<f_\mathrm{min}$ can be changed to an
interpolation by assuming a zero imaginary response at zero frequency, and we
then use PCHIP to interpolate the missing frequencies (as we work on a
logarithmic scale we cannot choose 0\,Hz, but instead take $10^{-100}$\,Hz). In
the case of the DLF method we also have to interpolate in between the computed
frequencies, for which we found it better to use a cubic spline. As can be seen
from Figures \ref{fig:GUI-FFTLog-lin} and \ref{fig:GUI-DLF-lin}, using the
FFTLog or the DLF with the same actually computed frequencies results in very
similar responses.

The changes to the Fourier transform in comparison with \cite{GEO.08.Mulder}
can be summarized in three points: (1) regular log-scale spacing for the
frequency selection; (2) DLF or FFTLog instead of FFT; and (3) using only the
imaginary part of the frequency-domain response. The actual speed of the
transform is unimportant, as the computation of the frequency-domain responses
takes much longer than the transform itself. What matters is solely how many
frequencies are required by it in order to be stable, and how long it takes to
compute the responses for these frequencies.

\section{Gridding}

The computation grid consists of a core or survey domain $D_\mr{s}$ which
should contain all source and receiver positions. The survey domain usually has
no or very little cell stretching $\alpha_\mr{s}$. The minimum cell width is
defined as
%
\begin{equation}
  \Delta_\mr{min}=\delta(f, \sigma_\mr{src})/n_\delta \, ,
  \label{eq:minwidth}
\end{equation}
%
where $\sigma_\mr{src}$ is the conductivity of the media in which the source
resides, and $n_\delta$ is a positive number that defines how many cells there
should be per skin depth. The actual computation domain $D_\mr{c}$ is usually
much bigger than $D_\mr{s}$ in order to avoid artefacts from the perfectly
electrically conducting boundary condition. It can also have a much higher
stretching $\alpha_\mr{c}$. In our scheme we have chosen $D_\mr{c}$ such that
the distance for the signal diffusing from the source to the boundary and back
to the receiver closest to the boundary is at least two wavelengths, after
which the initial signal is reduced to a millionth of its strength. The
wavelength to compute $D_\mr{c}$ is given by
%
\begin{equation}
  2\lambda = 4\pi\delta(f, \sigma_\mr{ave})
  \ \approx \ 6324.7/\sqrt{f\sigma_\mr{ave}}\, ,
 \label{eq:lambda}
\end{equation}
%
where $\sigma_\mr{ave}$ is the average conductivity, which can vary for
different directions. However, the skin-depth approach fails for air, in which
the EM field propagates as a wave at the speed of light. A largest
computational domain is therefore enforced, defining the maximum distance from
the source to the boundary; this distance is by default 100\,km, but this can
be reduced in the marine case with increasing water depth. Note that for
shallow marine and land cases this also applies to the horizontal dimensions,
not only to the upward $z$-direction and similarly for very resistive
basements, even in deep water. One way to circumvent this difficulty is the use
of a primary-secondary formulation, where the primary field, including the air
wave, is computed with a semi-analytical code for layered media. We do not
consider this approach here.

In summary, the adaptive gridding takes $f$, $D_\mr{s}$, $\sigma_\mr{src}$,
$\sigma_\mr{ave}$, $n_\delta$, and ranges for $\alpha_\mr{s}$, $\alpha_\mr{c}$,
where we usually fix $\alpha_\mr{s}=1$ or keep it at least below 1.05, and let
$\alpha_\mr{c}$ be anything between $[1, 1.5]$. The minimum cell width
$\Delta_\mr{min}$ can further be restricted by a user-defined range. Given
these inputs the adaptive gridding will search for the smallest possible number
of cells which fulfils these criteria. The multigrid method implemented in
\emg3d puts some constraints on the number of cells, of which the adaptive
gridding takes care (the number of cells have to be powers of two multiplied by
a low prime, e.g., $\{2,3,5\}\cdot2^n$).

The main difference with \cite{GEO.08.Mulder} is that their adaptive gridding
searches for the optimal stretching factor $\alpha$ fulfilling certain
criteria, for a fixed number of cells. Our adaptive gridding, on the other
hand, searches for the smallest number of cells that still fulfil the given
criteria. The number of cells becomes therefore also a function of frequency.
To go from the model grid to the computational grid, we use the
volume-averaging technique on logarithmic resistivities, as used in
\cite{GEO.07.Plessix}. While this technique ensures that the total resistivity
in the subsurface remains the same, it does not consider effective-medium
theory \citep{GEO.03.Davydycheva}, for instance, the apparent anisotropy from a
stack of finely layered formations of varying resistivity.

\section{Numerical examples}

\subsection{Homogeneous space}

The first example is the inline electric field from a source at the origin
measured by an inline receiver with an offset of 900\,m in a homogeneous space
of $1\,\Omega\,$m. We chose this simple example to compare it with the
analytical solution and with previously published results. We used the
following values to define the required frequencies: $f_\mr{min}=0.05\,$Hz,
$f_\mr{max}=21\,$Hz, using FFTLog with 5 frequencies per decade. This results
in 14 frequencies to compute from 0.05\,Hz to 20.0\,Hz. The complete frequency
range for the transform includes 30 frequencies from 0.0002\,Hz to 126.4\,Hz.
For the adaptive gridding the following inputs were used: $n_\delta=12$,
minimum cell width must be between 20 and 40\,m, and $\alpha_\mr{s}=1$,
$\alpha_\mr{c}=[1,1.3]$. This created grids with cell numbers between
\num{46080} ($80\times24\times24$, for 20.0\,Hz) and \num{128000}
($80\times40\times40$, for 0.05\,Hz) cells. The run times for each frequency,
the corresponding number of cells, minimum cell width and computation domain
stretching factor are listed in Table~\ref{tbl:timefull}. The total run time to
compute this model was less than two minutes.
%
\tabl[btp]{timefull}{Run times per frequency for the homogeneous space example,
  with the corresponding number of cells and minimum cell width as well as the
  stretching factor in the computation domain; $\alpha_\mr{s}=1$ everywhere.}{
  \centering
\begin{tabular}{S[table-format=2.4]rrrr}
  \toprule
  %
  {freq.} & time & nx$\times$ny$\times$nz & $\Delta_\mr{min}$ & $\alpha_\mr{c}$ \\
  {(Hz)}  & (s) &  & (m) & \\
  \midrule
  %
  20.0    &  3 & 80$\times$24$\times$24 & 20 & 1.26 \\
  12.6    &  6 & 96$\times$32$\times$32 & 20 & 1.17 \\
   7.98   &  8 & 96$\times$32$\times$32 & 20 & 1.20 \\
   5.03   &  7 & 96$\times$32$\times$32 & 20 & 1.23 \\
   3.18   &  7 & 80$\times$32$\times$32 & 24 & 1.25 \\
   2.00   &  5 & 80$\times$32$\times$32 & 30 & 1.21 \\
   1.26   &  4 & 64$\times$32$\times$32 & 37 & 1.21 \\
   0.798  &  5 & 64$\times$32$\times$32 & 40 & 1.23 \\
   0.503  &  5 & 64$\times$32$\times$32 & 40 & 1.26 \\
   0.318  &  5 & 64$\times$32$\times$32 & 40 & 1.28 \\
   0.200  & 10 & 64$\times$40$\times$40 & 40 & 1.27 \\
   0.126  & 10 & 64$\times$40$\times$40 & 40 & 1.30 \\
   0.0798 & 14 & 80$\times$40$\times$40 & 40 & 1.26 \\
   0.0503 & 15 & 80$\times$40$\times$40 & 40 & 1.28 \\
  %
  \bottomrule
\end{tabular}}%
%

Figure \ref{fig:fullspace} (a) shows the frequency-domain result, where the
blue dots are the computed responses and the black dots correspond to the
interpolated values or the values set to zero.
%
\plot*{fullspace}{width=\textwidth}{
  (a) Frequency- and (b) time-domain results for the homogeneous space model.
  The red lines are the analytical solutions, the blue circles are the actually
  computed responses with \emg3d, the black dots are the interpolated
  responses, and the dashed black line the obtained time-domain response. The
  errors are clipped for values smaller than 0.1\,\% and bigger than 100\,\%.
}
%
Most of the computed values stay below a relative error of 1\,\%, our
chosen adaptive gridding only starts to generate considerable errors at higher
frequencies. Figure \ref{fig:fullspace}(b) shows the corresponding time-domain
result, where the dashed black line is the result from \emg3d, on top of the
red line which is the analytical result. The relative error is mostly below
1\,\%, except for early times. However, for practical reasons that is more than
enough. Figure~\ref{fig:fullspace-log} shows the same on a logarithmic scale,
with times up to 10 seconds. It clearly shows that if later times are required,
we would need to adjust our Fourier transform parameters.
%
\plot{fullspace-log}{width=\cwidth}{
  Same as in Figure~\ref{fig:fullspace} (b), but on a logarithmic scale. To
  improve later times we would have to compute lower frequencies; to improve
  earlier times we would have to compute more frequencies per decade to get a
  better resolution.
}
%
Note that for the gridding we chose $n_\delta=12$, which is very dense. This
was necessary because we are relatively close to the source. If the offsets of
interest are larger this factor can be lowered considerably; 3--4 is often
enough.

This model corresponds to the one presented in Table 1 and in Figures 3 and 4
of \cite{GEO.08.Mulder}. The response here appears to be more accurate, their
reported peak-error is roughly 1\,\%, whereas we are below 0.1\,\% at the peak
(there are no error-plots presented, so visual inspection is all we have).
However, the difference in run time is dramatic. Summing the run times for the
different frequencies of the original figure comes to a total computation time
of 3\,h 47\,min 12\,s; 0.01\,Hz was the slowest run with 31\,min 19\,s, and
2.37\,Hz was the fastest run with 2\,min 54\,s. Our example, on the other hand,
took less than two minutes in total, where the individual frequencies took
between 3 and 15\,s to run.

This massive speed-up has a couple of reasons. Computers have become more
powerful in the last twelve years, and the codes were run on different
computers. A quick test with the old scripts on our test machine shows that it
would roughly run 2--3 times faster, therefore somewhere between 1 and 2 hours.
The more important facts besides different hardware are: (1) we only used 14
frequencies instead of the 26 frequencies between 0.01 and 100\,Hz of the
original; (2) our adaptive gridding used significantly less cells
($f$-dependent) in comparison to the fixed \num{2097152} cells ($128^3$) used
in the original example. We did not see a significant difference in the speed
of the actual codes, where the kernel-algorithm of the two implementations is
the same, but in the original example it is implemented in Matlab/C, whereas
\emg3d is written in Python/Numba (Numba is a just-in-time compiler for Python
code, \citealp{Numba}).

\subsection{1D Model}

The second example is a shallow marine, layered model with 200\,m of seawater
(3\,S/m) above a halfspace of 1\,S/m, an embedded target layer at 2\,km depth,
100\,m thick, with a conductivity of 0.02\,S/m. The source is located 20\,m
above the seafloor and the receivers are on the seafloor. We chose the
frequency range such that we can model offsets from 3 to 7 kilometers, with
$f_\mr{min}=0.007\,$Hz and $f_\mr{max}=32\,$Hz, using FFTLog with 5 frequencies
per decade. This results in computations for 19 frequencies from 0.008\,Hz to
31.8\,Hz. The complete frequency range for the transform includes 35
frequencies from $2\,10^{-5}$ to $126.4\,$Hz. For the adaptive gridding, we
used a cell width of 100\,m in the core domain and stretching outside up to a
factor 1.5, where the computation domain extends up to 50\,km in each
direction. This yielded grids between \num{204800} (higher frequencies) and
\num{245760} (lower frequencies) cells. The run times for each frequency and
their corresponding parameters are listed in Table~\ref{tbl:timemarine}.
%
\tabl[btp]{timemarine}{Run times per frequency for the marine 1D example,
  with the corresponding number of cells and minimum cell width as well as the
  stretching factor in the computation domain; $\alpha_\mr{s}=1$ everywhere.}{
  \centering
\begin{tabular}{S[table-format=2.5]rrrr}
  \toprule
  %
  {freq.} & time & nx$\times$ny$\times$nz & $\Delta_\mr{min}$ & $\alpha_\mr{c}$ \\
  {(Hz)}  & (s) &  & (m) &  \\
  \midrule
  %
  31.8     & 10 & 128$\times$40$\times$40 & 100 & 1.36 \\
  20.0     &  9 & 128$\times$40$\times$40 & 100 & 1.36 \\
  12.6     &  9 & 128$\times$40$\times$40 & 100 & 1.36 \\
   7.98    & 17 & 128$\times$40$\times$40 & 100 & 1.36 \\
   5.03    & 21 & 128$\times$40$\times$40 & 100 & 1.44 \\
   3.18    & 20 & 128$\times$40$\times$40 & 100 & 1.48 \\
   2.00    & 18 & 128$\times$40$\times$40 & 100 & 1.49 \\
   1.26    & 22 & 128$\times$40$\times$48 & 100 & 1.36 \\
   0.798   & 26 & 128$\times$40$\times$48 & 100 & 1.36 \\
   0.503   & 29 & 128$\times$40$\times$48 & 100 & 1.36 \\
   0.318   & 42 & 128$\times$40$\times$48 & 100 & 1.36 \\
   0.200   & 41 & 128$\times$40$\times$48 & 100 & 1.38 \\
   0.126   & 48 & 128$\times$40$\times$48 & 100 & 1.40 \\
   0.0798  & 54 & 128$\times$40$\times$48 & 100 & 1.41 \\
   0.0503  & 59 & 128$\times$40$\times$48 & 100 & 1.44 \\
   0.0318  & 74 & 128$\times$40$\times$48 & 100 & 1.44 \\
   0.0200  & 81 & 128$\times$40$\times$48 & 100 & 1.47 \\
   0.0126  & 96 & 128$\times$40$\times$48 & 100 & 1.48 \\
   0.00798 & 93 & 128$\times$40$\times$48 & 100 & 1.49 \\
  %
  \bottomrule
\end{tabular}}%
%

Figure~\ref{fig:marine} shows the result for an offset of 5\,km, in (a) the
frequency and (b) the time domain. The recovered response with the 3D code
captures the airwave (first peak) and the subsurface (second peak) very
accurately. At later times the error starts to increase. We would need to
compute a few additional lower frequencies if we want to improve it.
%
\plot*{marine}{width=\textwidth}{
  Electric inline response at an offset of 5\,km for a shallow marine, layered
  scenario. (a) Frequency-domain, where the blue circles are computed responses
  and the black dots are interpolated responses or responses set to zero. (b)
  Time-domain response.
}
%
In the frequency-domain plot, it can be seen that the high frequencies are not
computed very accurately, but without too much influence on the time-domain
response. These frequencies could be left out if an offset of 5\,km is the only
objective. However, we also want to retrieve shorter offsets from the same
computation, for which these frequencies are required.

Figure~\ref{fig:marine-multioffset} shows the time-domain responses of the same
model for offsets of 3, 5, and 7\,km, all obtained with the same
frequency-domain computations and the same frequencies for the Fourier
transform. The computation of these frequencies took less than 13 minutes and
it handles any offset between 3 and 7\,km. It can be seen that the chosen
frequency selection is sufficient for this offset range; again, more low
frequencies could be added to improve late-time values.
%
\plot{marine-multioffset}{width=\cwidth}{
  Time-domain responses for offsets of 3, 5, and 7\,km for the same model as
  shown in Figure~\ref{fig:marine}.
}
%

\subsection{Horizontal extent of the computation domain}

The skin-depth approach fails for the air layer, as explained in the Section
\emph{Gridding}. The reason is that the EM field in the air travels at the
speed of light as a wave, and its amplitude is only reduced through geometrical
spreading. On land and in shallow marine scenarios one has therefore to include
a sufficiently large computational domain, and the default in our scheme is
100\,km. The important point is that this does not only apply to the upward
$z$-direction, but also to the horizontal directions, as the airwave also
bounces back horizontally and would continuously emit energy into the
subsurface if the boundaries are not chosen far enough away from the receivers.
If models are computed with very resistive layers or models with highly
resistive basements, this can even apply to deep marine scenarios.

Figure~\ref{fig:marine-wrong-x-y} shows this effect. It is the same model as in
the previous section; however, for the adaptive gridding in the horizontal
directions $\rho_\mr{ave}=1\,\Omega\,$m was used instead of
$\rho_\mr{ave}=\num{10000}\,\Omega\,$m. Having the boundaries too near in the
horizontal directions leads to worse results for most frequencies and entirely
wrong results for high frequencies. Comparison with the 1D result in the time
domain shows that it is the airwave whose amplitude is heavily overestimated.
%
\plot*{marine-wrong-x-y}{width=\textwidth}{
  Same model as used for Figure~\ref{fig:marine}, but with the horizontal
  boundaries not far enough. Although the resulting time-domain result looks
  plausible, the comparison with the 1D result shows that it significantly
  overestimates the amplitude of the airwave.
}
%
It can be difficult to spot these errors in the time-domain result, as the
response looks plausible and only comparing it with the 1D result reveals that
it is actually wrong. A possibility to detect such problems for complicated
cases, where there is no semi-analytical result to compare with, is to compute
two or more models, moving the boundary. When the responses stop to change, one
can assume that the boundary is far enough. Another possibility is to look at
the amplitudes close to the boundaries and ensure that they are small enough.


\subsection{3D Model}

The final example consists of a resistive, three-dimensional block embedded in
the lower of two halfspaces, as depicted in Figure~\ref{fig:3d-model}. The
target has resistivity $\rho_\mr{tg}=100\,\Omega\,$m, the upper halfspace
corresponds to seawater with $\rho_\mr{sea}=0.3\,\Omega\,$m, and the lower
halfspace is the background with $\rho_\mr{bg}=1\,\Omega\,$m.
%
\plot{3d-model}{width=\cwidth}{
  Three-dimensional block embedded in the lower of two halfspaces. The 100-m
  long, $x$-directed dipole source is located 50\,m above the seafloor at the
  origin, and the receiver is on the seafloor at an offset of 2\,km.
}
%
The source is a 100-m long, $x$-directed dipole at the origin, 50\,m above the
seafloor, and we are using a step-off source function. The $x$-directed inline
receiver is at an offset of 2\,km. The dimension of the target cube is
$1.1\times1.0\times0.4$\,km, located 300\,m below the seafloor in the centre
between source and receiver.

For the comparison we use the open-source code \simpeg \citep{CAG.15.Cockett}.
\simpeg is a framework for modelling and inversion of geophysical data such as
gravity, magnetics, and CSEM. It has Maxwell's equations implemented in both
the frequency and time domain. As such we can compare our result computed in
the frequency domain followed by a Fourier transform to a result computed
directly in the time domain. A principal difference between \simpeg and \emg3d
is that the former has various direct solvers implemented, whereas the latter
is an iterative multigrid solver. The 3D model is therefore a rather small
example in order to be able to run it on our test machine, as the memory
requirement by the direct solver would otherwise be too high. There are not
many options out there of open-source time-domain 3D codes, \simpeg being the
one we found to be suitable. A step-off response was chosen as this is the
response currently implemented in \simpeg.

The model was discretised with $100\times100\times100\,$m cells in the survey
domain $D_\mr{s}$. For the \simpeg model, 14 cells in $x$-direction and 12
cells in $y$- and $z$-directions were used on both sides with a stretching of
1.3 for the total computation domain $D_\mr{c}$, which yields a mesh of
\num{58344} cells. The time-steps start at 0.1\,s and are: $21\times0.01\,$s,
$23\times0.03\,$s, $21\times0.1\,$s, $23\times0.3\,$s, covering exactly the
desired range of 0.1--10\,s. For \emg3d, the mesh is generated
frequency-dependent as in the previous examples, with a maximum stretching of
$\alpha_\mr{c}=1.5$. This results in meshes between \num{18432} cells for the
highest frequencies and \num{76800} cells for the lowest frequencies. The
required frequencies were obtained by using the FFTLog with five points per
decade, which results in 20 frequencies between 0.001\,Hz and 8\,Hz. The actual
transform was carried out with the 201-point sine-cosine filter from
\cite{GEO.09.Key}.

The results are shown in Figure~\ref{fig:3d-result}: In (a) the 1D background
responses and the relative error using the semi-analytical result obtained from
\empymod, and in (b) the responses including the target.
%
\plot{3d-result}{width=\textwidth}{
  Responses for the model outlined in Figure~\ref{fig:3d-model} using \simpeg
  and \emg3d, and for the layered background also \empymod. In the lower plot
  of (a) the relative error (\%) in comparison to \empymod is shown, and in (b)
  the normalized difference (\%) between the two 3D codes. The results of these
  plots are clipped for values smaller than 0.1\,\% and bigger than 10\,\%.
}
%
The background comparison shows that both 3D codes do an acceptable job having
a relative error of a few percents at most; the result for \emg3d seems to be
better at early times. We cannot compare the errors for the response that
includes the target for lack of an analytical solution. The background model
from \empymod is only included to show that there is a significant response
from the target. We therefore show the normalized difference (ND) between the
two 3D codes as a percentage, where $\mr{ND}=
50|(\simpeg-\emg3d)/(\simpeg+\emg3d)|$. The ND between the two codes is below
1\,\% everywhere except for early times.

Both codes took a little over 7 minutes to compute the two models (single
thread). However, in this particular comparison, the main difference in runtime
is not frequency-domain computation vs. time-domain computation, but iterative
solver vs. direct solver. For a test we doubled the number of cells in each
direction by using 50\,m for the minimum cell width instead of 100\,m, and
taking also half of the stretching, which results then in a mesh of
\num{466752} cells for \simpeg and meshes between \num{102400} and \num{614400}
for \emg3d. Because the memory of our test machine was not large enough for the
direct solver of \simpeg, the last model had to be run on a cluster. In this
scenario, \emg3d was about eight times faster (single thread).

\section{Conclusions}

We have shown a method to minimize the required frequencies and their range for
the computation of time-domain CSEM data with a frequency-domain code. This can
significantly reduce the computation time and makes time-domain CSEM modelling
with a frequency-domain code competitive given a robust frequency-domain
solver, a frequency-dependent gridding function that minimizes the required
cells, and a Fourier transform that works on a logarithmic scale. Fast layered
modelling can be used to design the required frequency range, as the Fourier
transform does not know about the dimensionality of the underlying model.
Twenty frequencies or less are usually sufficient for a wide range of offsets.
We believe that our proposed improvements to the previously published methods
make time-domain CSEM modelling with a frequency-domain code even more
attractive than it was already before. We have demonstrated the idea of our
Fourier transform method on CSEM data transformed from the frequency to the
time domain. However, it could equally be applied to the transform from the
time to the frequency domain and to other methods with similar
characteristics.

\section{Acknowledgment}

This research was conducted within the Gitaro.JIM project funded through
MarTERA as part of Horizon 2020, a funding scheme of the European Research Area
(ERA-NET Cofund, \href{https://www.martera.eu}{martera.eu}).
An early idea of this manuscript was presented at \cite{EAGE.20.Werthmuller}.

\section{Data Availability Statement}

The data that support the findings of this study are openly available at Zenodo
at \href{https://zenodo.org/badge/DOI/10.5281/zenodo.???????}{doi:
10.5281/zenodo.???????}. The data includes the scripts and instructions to
reproduce all results and figures.

\emph{Note to editors and reviewers: The data are NOT YET on Zenodo, we will
  upload there the final versions and link the right DOI for the final
  manuscript. Until then everything can be found on GitHub at
  \href{https://github.com/empymod/article-TDEM}%
  {github.com/empymod/article-TDEM}.}

% REFERENCES

\bibliographystyle{Refs}          % Modified SEG bibliography style.
\bibliography{Refs}

\end{document}
